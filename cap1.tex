\chapter{Introducción}

En esta sección, se comienza por introducir el contexto sobre el que se identificó el problema. A continuación, se describe la problemática que se pretende resolver y se presentan algunas de las soluciones que existen actualmente. Es decir, se explica cual es el estado del arte en el problema especificado. Finalmente, se plantea en términos generales el objetivo del presente trabajo y la manera en que será organizado.

\section{Contexto y problemática} 
A pesar de tener un mercado laboral robusto y dinámico, no existe en el mercado mexicano -y en general en el mundo- una solución moderna, económica y eficiente para conectar la demanda y la oferta de posiciones laborales. En la actualidad, en la economía mexicana existen alrededor de 2 millones de personas desempleadas \cite{derechohabiencia} y 60 millones de personas empleadas \cite{ocupacion}, de las cuales, más del 60\% se encuentran en la informalidad \cite{informalidad}. Al mismo tiempo, las empresas más grandes del país reportan no poder llenar sus vacantes con suficiente velocidad para crecer al ritmo que su potencial permitiría. Según datos del instituto mexicano del seguro social, en México se generan únicamente 100,000 empleos formales al mes1. Es decir, hay personas en busca de empleo y empresas en busca de talento y no se están encontrando. 

Es importante precisar que si bien este fenómeno es particularmente agudo en el mercado laboral mexicano, se encuentra presente en mayor o menor medida en todos los mercados laborales del mundo.

De lo anterior resulta evidente que la asimetría de información en el mercado laboral es un problema relevante y a través del cual existe la oportunidad de generar mucho valor e impactar a un gran número de personas.

En la actualidad, los dos principales participantes del mercado laboral -empleados y empleadores- están tomando decisiones con información incompleta y poco confiable. Este fenómeno es particularmente pronunciado en los puestos de entrada y tiene diversos efectos negativos sobre el desarrollo de las actividades de las empresas y el desempeño de las trabajadoras en las mismas.

Por un lado, las empresas tienen la necesidad de llenar un gran número de vacantes con mucha regularidad y en un horizonte temporal bastante reducido. Lo anterior, resulta en un proceso de selección y contratación poco exhaustivo y que apunta a contratar a un número elevado de personas sin mayor valoración y evaluar su idoneidad para el puesto una vez que ya se encuentran desempeñándolo. Esto tiene importantes repercusiones negativas sobre los resultados de las empresas, pues no solo resulta más costoso debido a los costos asociados al proceso de reclutamiento y contratación. La productividad de las empresas se ve también afectada, pues como consecuencia de la falta de alineación entre las habilidades y experiencia de las aspirantes contratadas y la requerida por el puesto, un gran número de ellas muestran niveles de desempeño por debajo de lo mínimo requerido para la posición. Esto se puede ver reflejado en casos tan variados como una empleada de primera línea con actitud negativa atendiendo pobremente a los clientes de la empresa, o una obrera en una línea de producción cometiendo errores, que debido a una insuficiente comprensión de los sistemas utilizados, podrían poner en riesgo la confiabilidad de los productos de la empresa.

Por el otro lado, las personas en busca de un empleo, se encuentran en una situación aún más complicada. Se enfrentan a dos principales dificultades que les impiden acceder a las vacantes que maximizan el aprovechamiento de sus habilidades y experiencia: información incompleta/asimétrica y ausencia de recursos de señalización/diferenciación.

La primera de estas dificultades se manifiesta en la información incompleta y asimétrica a la que el candidato promedio tiene acceso. Mientras las empresas tienen presupuesto para departamentos enteros de recursos humanos que les permiten tener una visión y alcance relativamente amplios del mercado en su totalidad, las aspirantes no cuentan con los recursos y/o el tiempo para descubrir y evaluar las miles de vacantes para las que califican. Dado lo anterior, las personas en busca de empleo, suelen aceptar el primer empleo que les es ofrecido sin haber evaluado si es la vacante que mejor se ajusta a sus intereses, habilidades y experiencia. Es decir, aceptan lo que se les ofrece sin mayor consideración -lo que en la ciencia económica se conoce como “price taker”.
La segunda instancia en la que las personas en busca de empleo se encuentran en desventaja, se relaciona con su incapacidad para señalizar al mercado laboral sobre aquellas cosas que los hacen diferentes y/o mejores que el resto de las aspirantes a una vacante dada. El mercado laboral en la actualidad, ya requiere de ciertas “señales estándar” por parte de las personas en busca de empleo. Algunas de estas señalizaciones aceptadas son: credenciales académicas, cartas de recomendación, exámenes psicométricos, etc. El problema con las señalizaciones existentes es que su confiabilidad se ve comprometida debido a la alta variabilidad en su desempeño como predictores de éxito futuro en un puesto laboral particular. 

Como ejemplo para ilustrar esta dinámica, podemos imaginar una empresa que utiliza un título universitario como único filtro de selección para sus vacantes. Dentro de las empleadas que contrató utilizando este criterio, la empresa evalúa su desempeño después de 6 meses de trabajo en una escala del 1 al 10; obteniendo resultados desde los 3 y hasta los 9 puntos. Suponiendo que la empresa iniciara una nueva ronda de contratación en la cual buscara una calificación mínima de 5 para poder acceder al puesto, el filtro universitario no podría ser contemplado como una opción viable para obtener los resultados esperados, debido a que este solo garantiza que las aspirantes tendrán resultados iguales o superiores a 3. Dado lo anterior, una graduada universitaria que se sabe con capacidad 9, no tiene manera de demostrar al mercado cuál es la diferencia entre ella y otra graduada universitaria con capacidad 3 y la inversión que hizo para obtener esa señalización se vuelve obsoleta.

En resumen, bajo las condiciones actuales, el mercado laboral es sumamente ineficiente y dado que sus agentes principales resultan afectados por estas ineficiencias, existen incentivos para aceptar soluciones novedosas que permitan una mejor asignación del capital humano en el mercado laboral.




\section{Objetivos del trabajo}
Este trabajo pretende proponer una solución novedosa al problema de asimetría de información en el mercado laboral incorporando algunas de las características positivas de las soluciones ya existentes. El núcleo de la solución propuesta es el desarrollo de una plataforma estilo “marketplace” que tenga la capacidad de evaluar la compatibilidad de cada candidato registrado con cada vacante anunciada. Es decir, que pueda generar una calificación de “compatibilidad” entre cada pareja vacante-aspirante y permita con ella, mostrar a cada usuario de la plataforma una lista personalizada con las opciones más relevantes para su caso particular.
En función de lo anterior, el objetivo del presente trabajo es contribuir al desarrollo de un mercado laboral más eficiente a través de una mejor asignación del capital humano disponible. En otras palabras, permitir que cada persona trabaje en la posición en donde pueda generar más valor y desarrollarse al máximo de su potencial. 
\section{Organización del documento}
En el primer capítulo, se introduce la problemática y se analizan las razones por las que tiene sentido trabajar sobre este problema.
En el segundo capitulo, se describen los requerimientos funcionales que se requieren para cumplir con los objetivos del trabajo. En este capítulo, también se explica cuales son las restricciones a las que se tuvo que ajustar el diseño de la solución.
El tercer capítulo expone el diseño de la solución propuesta. Comienza por describir algunas soluciones alternativas que se consideraron. Posteriormente, se describe detalladamente el diseño y todos sus componentes. El capítulo concluye con una explicación de los estándares elegidos en el diseño.
En el cuarto capítulo, se describe la implementación que se hizo, explicando los pasos que se siguieron y y las dificultades que se encontraron.
El quinto capítulo aborda las pruebas que se realizaron sobre la implementación y los resultados que se obtuvieron de las mismas. El capítulo concluye con un análisis detallado de las implicaciones de los resultados obtenidos.
El sexto y último capítulo es la conclusión. En ella se analizan las repercusiones del trabajo y los posibles siguientes pasos para la solución propuesta.